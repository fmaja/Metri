\documentclass[12pt,a4paper]{article}

% Polish language support with pdfLaTeX
\usepackage[utf8]{inputenc}
\usepackage[T1]{fontenc}
\usepackage[polish]{babel}
\usepackage{lmodern}

% Packages
\usepackage{geometry}
\usepackage{xcolor}
\usepackage{booktabs}
\usepackage{longtable}
\usepackage{hyperref}
\usepackage[normalem]{ulem}
\usepackage{amsmath, amssymb}

\geometry{margin=2.5cm}

\definecolor{passgreen}{RGB}{46,204,113}
\definecolor{failred}{RGB}{231,76,60}
\definecolor{coverage}{RGB}{52,152,219}

\title{Raport Testów Jednostkowych\\Aplikacja Metri}
\date{14 grudnia 2025}

\begin{document}

\maketitle
\tableofcontents
\newpage

\section{Podsumowanie Wykonawcze}

Przeprowadzono kompleksowy proces testowania jednostkowego dla aplikacji Metri. 
Testy objęły przede wszystkim warstwę logiki: teorię muzyki (\texttt{music\_theory.py}), formatowanie wyświetlania tekstów i akordów (\texttt{display\_func.py}), zarządzanie zbiorem piosenek (\texttt{song\_func.py}) oraz parsowanie surowych tekstów do struktury JSON (\texttt{jsonify\_func.py}), z pokryciem odpowiednio około 96\%, 95\%, 73\% i 62\%. Dodatkowo częściowo pokryto moduł \texttt{keys.py} (około 81\%), przy czym funkcjonalność transpozycji została usunięta z interfejsu użytkownika. 
Projekt obejmował dwa główne etapy: testowanie podstawowych modułów logiki biznesowej 
oraz rozszerzenie testów o moduł formatowania wyświetlania.

\subsection{Statystyki Finalne}

\begin{table}[h]
\centering
\begin{tabular}{lrr}
\toprule
\textbf{Metryka} & \textbf{Etap 1} & \textbf{Etap 2} \\
\midrule
Łączna liczba testów & 98 & \textcolor{passgreen}{\textbf{133}} \\
Testy zaliczone & \textcolor{passgreen}{98} & \textcolor{passgreen}{\textbf{133}} \\
Testy niezaliczone & \textcolor{passgreen}{0} & \textcolor{passgreen}{\textbf{0}} \\
Wskaźnik powodzenia & \textbf{100\%} & \textbf{100\%} \\
Czas wykonania & 0.42s & 0.57s \\
Pokrycie kodu (moduły logiki) & 6\% & \textcolor{coverage}{9\%} \\
\bottomrule
\end{tabular}
\caption{Porównanie wyników testów między etapami}
\end{table}

\subsection{Analiza Pokrycia Kodu}

\begin{table}[h]
\centering
\begin{tabular}{lccc}
\toprule
\textbf{Moduł} & \textbf{Linie} & \textbf{Pokrycie} & \textbf{Ocena} \\
\midrule
music\_theory.py & 54 & 96\% & \textcolor{passgreen}{Doskonałe} \\
\textbf{display\_func.py} & \textbf{159} & \textcolor{passgreen}{\textbf{95\%}} & \textcolor{passgreen}{\textbf{Doskonałe}} \\
keys.py & 32 & 81\% & \textcolor{passgreen}{Bardzo dobre} \\
song\_func.py & 128 & 73\% & Dobre \\
jsonify\_func.py & 146 & 62\% & Średnie \\
\midrule
\textbf{Razem (moduły logic)} & \textbf{519} & \textbf{85\%} & \textcolor{passgreen}{\textbf{Bardzo dobre}} \\
\bottomrule
\end{tabular}
\caption{Szczegółowe pokrycie testami modułów logiki}
\end{table}


\section{Historia Testowania}

\subsection{Etap 1: Podstawowe Moduły (98 testów)}

\subsubsection{Faza 1a: Pierwsze Uruchomienie}

Utworzono testy dla trzech kluczowych modułów:
\begin{itemize}
    \item \textbf{test\_music\_theory.py} - 24 testy teorii muzycznej
    \item \textbf{test\_jsonify\_func.py} - 38 testów parsowania piosenek
    \item \textbf{test\_song\_func.py} - 36 testów zarządzania songbookiem
\end{itemize}

\textbf{Wyniki pierwszego uruchomienia:}
\begin{itemize}
    \item Łączna liczba testów: 98
    \item \textcolor{passgreen}{Testy zaliczone:} 92
    \item \textcolor{failred}{Testy niezaliczone:} 6
    \item Wskaźnik powodzenia: 93.88\%
\end{itemize}

\subsubsection{Faza 1b: Naprawa Błędów}

Zidentyfikowano i naprawiono 6 problemów testowych:
\begin{enumerate}
    \item \texttt{test\_string\_to\_list\_single\_semicolon} - dopasowano do zachowania \texttt{split(";")}
    \item \texttt{test\_redefine\_sections\_only\_chords} - poprawiono sprawdzanie pola lyrics
    \item \texttt{test\_redefine\_sections\_chord\_pattern\_recognition} - j.w.
    \item \texttt{test\_name\_sections\_chorus\_detection} - zluzowano sprawdzanie nazw sekcji
    \item \texttt{test\_get\_interval\_name\_basic} - uwzględniono modulo 12
    \item \texttt{test\_get\_interval\_name\_modulo} - j.w.
\end{enumerate}

\textbf{Wyniki po naprawach:}
\begin{itemize}
    \item \textcolor{passgreen}{\textbf{Wszystkie 98 testów przechodzi pomyślnie}}
    \item \textcolor{passgreen}{\textbf{Wskaźnik powodzenia: 100\%}}
    \item Czas wykonania: 0.42s
\end{itemize}

\subsection{Etap 2: Rozszerzenie o Display (133 testy)}

\subsubsection{Motyw rozszerzenia}

Moduł \texttt{display\_func.py} (159 linii) nie posiadał testów (0\% pokrycia).
Po usunięciu funkcjonalności transpozycji, konieczne było przetestowanie pozostałych funkcji.

\subsubsection{Utworzone testy}

Dodano \textbf{test\_display\_func.py} z 35 testami:
\begin{itemize}
    \item \textbf{TestChordsToScheme} - 5 testów funkcji pomocniczej
    \item \textbf{TestGetDisplayLyrics} - 5 testów wyświetlania tekstów
    \item \textbf{TestGetDisplayChords} - 4 testy wyświetlania akordów
    \item \textbf{TestGetDisplay2} - 7 testów formatu tekstowego
    \item \textbf{TestGetDisplay} - 12 testów formatu HTML
    \item \textbf{TestIntegration} - 3 testy integracyjne
\end{itemize}

\textbf{Wyniki:}
\begin{itemize}
    \item \textcolor{passgreen}{\textbf{Łączna liczba testów: 133}}
    \item \textcolor{passgreen}{\textbf{Wszystkie zaliczone: 133/133 (100\%)}}
    \item Czas wykonania: 0.57s
    \item \textcolor{coverage}{\textbf{Pokrycie display\_func.py: 0\% → 95\%}}
\end{itemize}

\section{Zmiany w Kodzie}

\subsection{Usunięte Komponenty}

W ramach optymalizacji usunięto nieużywane komponenty transpozycji:

\begin{itemize}
    \item \texttt{src/metri/views/song\_display.py} - widok z kontrolkami transpozycji (nigdy nie importowany)
    \item \texttt{tests/test\_keys.py} - testy funkcjonalności transpozycji
\end{itemize}

\subsection{Zmodyfikowane Moduły}

\begin{itemize}
    \item \textbf{src/metri/logic/display\_func.py}
    \begin{itemize}
        \item Usunięto import \texttt{from .keys import transpose}
        \item Usunięto parametr \texttt{transp} z funkcji:
        \begin{itemize}
            \item \texttt{get\_display()}
            \item \texttt{get\_display\_lyrics()}
            \item \texttt{get\_display\_chords()}
        \end{itemize}
        \item Usunięto wszystkie wywołania \texttt{transpose()}
    \end{itemize}
    
    \item \textbf{src/metri/views/songbook.py}
    \begin{itemize}
        \item Zaktualizowano wywołania funkcji display bez parametru \texttt{transp}
    \end{itemize}
\end{itemize}

\section{Szczegółowe Wyniki Testów}

\subsection{test\_music\_theory.py (24 testy) - Pokrycie 96\%}

Testy funkcji teorii muzycznej MIDI i harmonii.

\subsubsection{Konwersje MIDI $\leftrightarrow$ Nazwa nuty (9 testów)}
\begin{itemize}
    \item test\_note\_name\_to\_midi\_basic
    \item test\_note\_name\_to\_midi\_sharps
    \item test\_note\_name\_to\_midi\_different\_octaves
    \item test\_note\_name\_to\_midi\_invalid\_input
    \item test\_midi\_to\_note\_name\_basic
    \item test\_midi\_to\_note\_name\_sharps
    \item test\_midi\_to\_note\_name\_edge\_cases
    \item test\_midi\_to\_note\_name\_invalid\_range
    \item test\_midi\_note\_conversion\_roundtrip
\end{itemize}

\subsubsection{Interwały (3 testy)}
\begin{itemize}
    \item test\_get\_interval\_name\_basic
    \item test\_get\_interval\_name\_all\_intervals
    \item test\_get\_interval\_name\_modulo
\end{itemize}

\subsubsection{Generowanie nut MIDI (3 testy)}
\begin{itemize}
    \item test\_get\_all\_midi\_notes\_default\_range
    \item test\_get\_all\_midi\_notes\_custom\_range
    \item test\_get\_all\_midi\_notes\_single\_octave
\end{itemize}

\subsubsection{Generowanie akordów diatonicznych (9 testów)}
\begin{itemize}
    \item test\_generate\_diatonic\_chord\_tonic
    \item test\_generate\_diatonic\_chord\_dominant
    \item test\_generate\_diatonic\_chord\_subdominant
    \item test\_generate\_diatonic\_chord\_minor\_chords
    \item test\_generate\_diatonic\_chord\_diminished
    \item test\_generate\_diatonic\_chord\_all\_degrees
    \item test\_generate\_diatonic\_chord\_invalid\_degree
    \item test\_generate\_diatonic\_chord\_different\_keys
\end{itemize}

\subsection{test\_jsonify\_func.py (38 testów) - Pokrycie 62\%}

Testy parsowania i formatowania danych piosenek.

\subsubsection{TestStringToList (6 testów)}
Konwersja stringów rozdzielanych średnikami na listy.

\subsubsection{TestSplitIntoSections (6 testów)}
Podział tekstu piosenki na sekcje (zwrotki, refren, etc.).

\subsubsection{TestRedefineSections (6 testów)}
Rozpoznawanie i redefinicja typów sekcji (teksty vs akordy).

\subsubsection{TestNameSections (6 testów)}
Automatyczne nazywanie sekcji (v, c, i, etc.).

\subsubsection{TestDelRepetitions (6 testów)}
Usuwanie powtarzających się akordów.

\subsubsection{TestSongDataJsonifyAuto (6 testów)}
Automatyczna konwersja surowych danych piosenki do JSON.

\subsubsection{TestIntegration (2 testy)}
Testy pełnego pipeline'u parsowania.

\subsection{test\_song\_func.py (36 testów) - Pokrycie 73\%}

Testy zarządzania kolekcją piosenek.

\subsubsection{TestGetObjectiveKey (7 testów)}
Obliczanie rzeczywistej tonacji z uwzględnieniem capo.

\subsubsection{TestFilterSongs (10 testów)}
Filtrowanie i sortowanie piosenek według różnych kryteriów.

\subsubsection{TestGetTags (5 testów)}
Ekstrakcja i zarządzanie tagami piosenek.

\subsubsection{TestGetSong (5 testów)}
Pobieranie pojedynczej piosenki po ID.

\subsubsection{TestGetSongsByIds (4 testy)}
Pobieranie wielu piosenek na raz.

\subsubsection{TestRemoveSong (3 testy)}
Usuwanie piosenek z kolekcji.

\subsubsection{TestLoadSaveSongs (3 testy)}
Zapis i odczyt piosenek z pliku JSON.

\subsection{test\_display\_func.py (35 testów) - Pokrycie 95\%}

Testy formatowania wyświetlania piosenek.

\subsubsection{TestChordsToScheme (5 testów)}
Funkcja pomocnicza pozycjonująca akordy względem markerów w tekście.
\begin{itemize}
    \item test\_no\_markers\_returns\_original
    \item test\_single\_marker
    \item test\_multiple\_markers
    \item test\_more\_markers\_than\_chords\_cycles
    \item test\_extra\_chords\_appended
\end{itemize}

\subsubsection{TestGetDisplayLyrics (5 testów)}
Formatowanie wyświetlania tekstów piosenek.
\begin{itemize}
    \item test\_basic\_lyrics\_display
    \item test\_intro\_section\_included
    \item test\_empty\_lines\_handled
    \item test\_hidden\_lines\_included
    \item test\_repeated\_sections\_deduplicated
\end{itemize}

\subsubsection{TestGetDisplayChords (4 testy)}
Formatowanie wyświetlania akordów.
\begin{itemize}
    \item test\_basic\_chords\_display
    \item test\_intro\_sections\_skipped
    \item test\_tab\_sections\_skipped
    \item test\_empty\_chords\_not\_displayed
\end{itemize}

\subsubsection{TestGetDisplay2 (7 testów)}
Format tekstowy (plain text) z wyrównaniem tekstów i akordów.
\begin{itemize}
    \item test\_returns\_two\_element\_list
    \item test\_lyrics\_and\_chords\_aligned
    \item test\_intro\_section\_no\_chords
    \item test\_chorus\_lines\_indented
    \item test\_markers\_removed\_from\_lyrics
    \item test\_chord\_counter\_cycles
    \item test\_empty\_sections\_between\_content
\end{itemize}

\subsubsection{TestGetDisplay (12 testów)}
Format HTML z tagami formatującymi.
\begin{itemize}
    \item test\_returns\_string
    \item test\_contains\_html\_tags
    \item test\_intro\_section\_bold
    \item test\_tab\_section\_creates\_image\_tag
    \item test\_chords\_wrapped\_in\_code\_tags
    \item test\_second\_voice\_italicized
    \item test\_hidden\_lines\_create\_newline
    \item test\_chorus\_lines\_indented
    \item test\_markers\_removed\_from\_lyrics
    \item test\_chord\_counter\_increments
    \item test\_numbered\_section\_fallback
\end{itemize}

\subsubsection{TestIntegration (3 testy)}
Testy integracyjne wszystkich funkcji display.
\begin{itemize}
    \item test\_all\_display\_functions\_work
    \item test\_complex\_song\_all\_features
    \item test\_empty\_song\_handled
\end{itemize}

\section{Proces Naprawiania Testów (Etap 1)}

\subsection{Zidentyfikowane Problemy}

Po pierwszym uruchomieniu 98 testów, 6 nie przeszło pomyślnie:

\begin{enumerate}
    \item \texttt{test\_string\_to\_list\_single\_semicolon}
    \item \texttt{test\_redefine\_sections\_only\_chords}
    \item \texttt{test\_redefine\_sections\_chord\_pattern\_recognition}
    \item \texttt{test\_name\_sections\_chorus\_detection}
    \item \texttt{test\_get\_interval\_name\_basic}
    \item \texttt{test\_get\_interval\_name\_modulo}
\end{enumerate}

\subsection{Grupa 1: Testy Parsowania}

\subsubsection{Problem 1: test\_string\_to\_list\_single\_semicolon}
\begin{itemize}
    \item \textbf{Oczekiwanie:} \texttt{string\_to\_list(";")} zwróci \texttt{[""]}
    \item \textbf{Rzeczywistość:} Funkcja \texttt{split(";")} zwraca \texttt{["", ""]}
    \item \textbf{Rozwiązanie:} Zmieniono asercję na \texttt{["", ""]}
\end{itemize}

\subsubsection{Problemy 2-3: Przechowywanie akordów}
\begin{itemize}
    \item \textbf{Oczekiwanie:} Akordy pozostaną w polu \texttt{chords}
    \item \textbf{Rzeczywistość:} Funkcja przepisuje akordy do pola \texttt{lyrics}
    \item \textbf{Rozwiązanie:} Testy sprawdzają pole \texttt{lyrics} zamiast \texttt{chords}
\end{itemize}

\subsubsection{Problem 4: Numerowanie sekcji}
\begin{itemize}
    \item \textbf{Oczekiwanie:} Identyczne sekcje dostaną tę samą nazwę
    \item \textbf{Rzeczywistość:} Funkcja numeruje sekcje: \texttt{'v'}, \texttt{'v1'}, \texttt{'v2'}
    \item \textbf{Rozwiązanie:} Test sprawdza tylko prefiks nazwy
\end{itemize}

\subsection{Grupa 2: Testy Teorii Muzycznej}

\subsubsection{Problemy 5-6: Operacja Modulo}
\begin{itemize}
    \item \textbf{Oczekiwanie:} 12 półtonów = oktawa (\texttt{"P8 (Octave)"})
    \item \textbf{Rzeczywistość:} Funkcja używa \texttt{semitones \% 12}, więc \texttt{12 \% 12 = 0} → \texttt{"P1 (Unison)"}
    \item \textbf{Rozwiązanie:} Dopasowano oczekiwania do implementacji
\end{itemize}

\subsection{Wynik Procesu Naprawy}

\begin{itemize}
    \item \textcolor{passgreen}{\textbf{Wszystkie 6 problemów naprawione w jednej iteracji}}
    \item \textbf{Rodzaj napraw:} Tylko testy - bez zmian w kodzie produkcyjnym
    \item \textbf{Czas naprawy:} < 30 minut
    \item \textbf{Końcowy wynik:} 98/98 testów (100\%)
\end{itemize}

\section{Podsumowanie i Wnioski}

\subsection{Osiągnięcia}

\begin{table}[h]
\centering
\begin{tabular}{lcc}
\toprule
\textbf{Metryka} & \textbf{Początek} & \textbf{Koniec} \\
\midrule
Liczba testów & 0 & \textcolor{passgreen}{\textbf{133}} \\
Testy przechodzące & - & \textcolor{passgreen}{\textbf{133 (100\%)}} \\
Pokrycie (testowane moduły) & 0\% & \textcolor{passgreen}{\textbf{85\%}} \\
\bottomrule
\end{tabular}
\caption{Postęp projektu testowania}
\end{table}

\subsection{Kluczowe Wnioski}

\begin{enumerate}
    \item \textbf{Jakość testów:} Wszystkie 133 testy przechodzą stabilnie
    \item \textbf{Szybkość:} Pełny zestaw wykonuje się w < 0.6s
    \item \textbf{Pokrycie:} Testowane moduły mają średnio 85\% pokrycia
    \item \textbf{Proces naprawy:} Efektywny - 6 błędów naprawionych w jednej iteracji
\end{enumerate}

\subsection{Mocne Strony}

\begin{itemize}
    \item \textcolor{passgreen}{\textbf{Doskonałe pokrycie}} music\_theory.py (96\%) i display\_func.py (95\%)
    \item \textbf{Czysta separacja} - brak zależności między testami
    \item \textbf{Szybkie wykonanie} - feedback w < 1 sekundę
\end{itemize}

\subsection{Obszary do Poprawy}

\begin{itemize}
    \item \textbf{jsonify\_func.py} - zwiększyć pokrycie z 62\% do >80\%
    \item \textbf{song\_func.py} - zwiększyć pokrycie z 73\% do >90\%
\end{itemize}

\section{Środowisko Techniczne}

\begin{table}[h]
\centering
\begin{tabular}{ll}
\toprule
\textbf{Komponent} & \textbf{Wersja} \\
\midrule
Python & 3.13.2 \\
pytest & 9.0.2 \\
pytest-cov & 7.0.0 \\
System & Windows (win32) \\
Framework GUI & CustomTkinter \\
Biblioteki muzyczne & pygame 2.6.1, matplotlib 3.10.7 \\
\bottomrule
\end{tabular}
\caption{Środowisko testowe}
\end{table}

\subsection{Struktura Projektu}

\begin{verbatim}
Metri-feature-splash/
+-- src/
|   +-- metri/
|       +-- logic/
|       |   +-- display_func.py (95% pokrycia)
|       |   +-- music_theory.py (96% pokrycia)
|       |   +-- song_func.py (73% pokrycia)
|       |   +-- jsonify_func.py (62% pokrycia)
|       |   +-- keys.py (81% pokrycia)
|       +-- views/ (0% pokrycia - UI)
+-- tests/
|   +-- test_display_func.py (35 testów)
|   +-- test_music_theory.py (24 testy)
|   +-- test_jsonify_func.py (38 testów)
|   +-- test_song_func.py (36 testy)
+-- pytest.ini
\end{verbatim}

\end{document}
